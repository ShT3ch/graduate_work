\documentclass[notes,compress]{beamer}
\mode<presentation>
{
  \usepackage[bars]{beamerthemetree}
  \beamertemplatetransparentcovereddynamic
  \beamertemplateballitem
}
\newcommand\Wider[2][3em]{%
\makebox[\linewidth][c]{%
\begin{minipage}{\dimexpr\textwidth+#1\relax} \raggedright#2 \end{minipage}%
}%
}

\usepackage{graphicx}


\usepackage{tikz}
\usetikzlibrary{shapes.geometric, arrows, decorations.markings}
\tikzstyle{rect} = [rectangle, minimum width=4cm, minimum height=2cm, text centered, draw=black, fill=orange!30]
\tikzstyle{arrow}=[draw, -latex]
\tikzstyle{mnode} = [circle,minimum size=5mm,fill=blue!20,draw,font=\sffamily\Large\bfseries, node distance=5cm]


\usepackage{pgf,pgfarrows,pgfnodes,pgfautomata,pgfheaps,pgfshade}
\usepackage{amsmath,amssymb}
%\usepackage[latin1]{inputenc}
\usepackage[utf8x]{inputenc}
\usepackage{xcolor}
\usepackage[english, russian]{babel}
\usetheme{Antibes}
\usecolortheme{dolphin}
\usepackage{beamerfoils}
%\usepackage{lmodern}
%\usepackage{beamerthemeshadow}
%\usepackage{beamerthemeplain}
\usepackage{beamerthemeclassic}
%\beamertemplateshadingbackground{red!7}{blue!7}
%\colorlet{beamerstructure}{purple!30}

\usepackage[T1]{fontenc}
\beamertemplatetransparentcovered

\title
      {{Вывод и решение уравнений с учетом случайных возмущений.}}
 \author{Студентка группы МК-410202 \\
 Сайфуллина Айгуль Ильгамовна \newline
 \newline
Научный руководитель: профессор, доктор физико-математических наук \\
Мельникова Ирина Валерьяновна \\}
 \institute{Уральский федеральный университет}
 %\subject{Theoretical Computer Science}
 \date{Екатеринбург,~2015}

 %\pgfdeclaremask{usu}{logousumask}

 \pgfdeclareimage[%mask=usu,
                 width=5cm]{usu-logo}{logousu}

 \logo{\vbox{\hbox to 1cm{\hfil\pgfuseimage{usu-logo}}}}


 \begin{document}

% \beamerboxesdeclarecolorscheme{alert}{purple!15}{white}

%
\frame{\titlepage}


 \LogoOff

%=========================================================
%=========================================================




\begin{frame}

{\textbf{Часть 0}} описание принятой модели \newline

{\textbf{Часть 1}} вывод уравнения вероятностных характеристик \newline

{\textbf{Часть 2}} вывод и решение стохастического уравнения \newline

{\textbf{Часть 3}} связь между уравнениями для вероятностных характеристик и стохастическими уравнениями

\end{frame}

\begin{frame}
    \frametitle{Рассмотрим систему, которая может находиться в двух состояниях в каждый момент времени $t$.}

    \begin{tikzpicture}[node distance=3cm, overlay]

        \node at (current page.west) (s_1) [rect,xshift=24mm, yshift = -51mm ] {$S_1(t)$};
        \node (s_2) [rect, right of=s_1, xshift=3cm] {$S_2(t)$};



        %\path[every node/.style={sloped,anchor=south,auto=false}, line width=2pt]
        %    (s_1) edge [loop above] node {$(\lambda,0)$} (s_1)
        %    (s_1) edge [<->, bend right] node {an} (s_2);
        %\draw[->, ultra thick] (s_1.south) -- ([yshift = -25mm]s_1.south);

        \only<2-2>{
                \node (1Node) [mnode, right of=s_1, xshift=-40mm, yshift = 14mm] {\small1};
                \node (2Node) [mnode, left of=s_2, xshift=40mm, yshift = 14mm] {\small2};

                \path[every node/.style={sloped,anchor=south,auto=false}, line width=2pt]
                    (s_1) edge [->, loop above] node {$[0,0]$} (s_1)
                    (s_2) edge [->, loop above] node {$[0,0]$} (s_2);
        }

        \only<3-3>{
                \node (3Node) [mnode, below left of=s_1,xshift=25mm] {\small3};
                \node (4Node) [mnode, below right of=s_1, xshift=-25mm] {\small4};

                \node (5Node) [mnode, below left of=s_2,xshift=25mm] {\small5};
                \node (6Node) [mnode, below right of=s_2, xshift=-25mm] {\small6};

                \path[every node/.style={sloped,anchor=south,auto=false}, line width=2pt]
                    (s_1) edge [->] node {$[-\lambda_1,]$} (3Node)
                    (s_1) edge [<-] node {$[\lambda_1,0]$} (4Node)
                    (s_2) edge [->] node {$[0,-\lambda_2]$} (5Node)
                    (s_2) edge [<-] node {$[0,\lambda_2]$} (6Node);
        }

        \only<4-4>{
                \node (7Node) [mnode, right of=s_1, xshift=-20mm, yshift = 15mm] {\small7};
                \node (8Node) [mnode, right of=s_1, xshift=-20mm, yshift = -15mm] {\small8};

                \path[every node/.style={sloped,anchor=south,auto=false}, line width=2pt]
                    ([yshift=4mm]s_1.east) edge [->] node {$[-\lambda_1,\lambda_2]$} ([yshift=4mm]s_2.west)
                    ([yshift=-8mm]s_1.east) edge [<-] node {$[\lambda_1,-\lambda_2]$} ([yshift=-8mm]s_2.west);
        }

        \only<5-5>{
                \node (10Node) [mnode, right of=s_1, xshift=-20mm, yshift = 31mm, label={[label distance=1cm]-90:$[\lambda_1,\lambda_2]$}] {\footnotesize10};
                \node (9Node) [mnode, right of=s_1, xshift=-20mm, yshift = -31mm, label={[label distance=1cm]90:$[-\lambda_1,-\lambda_2]$}] {\small9};

                \path[every node/.style={sloped,anchor=south,auto=false}, line width=2pt]
                    (s_1) edge [<-] node {} (10Node)
                    (s_2) edge [<-] node {} (10Node)
                    (s_1) edge [->] node {} (9Node)
                    (s_2) edge [->] node {} (9Node);
        }
    \end{tikzpicture}
\end{frame}

\begin{frame}
Уравнение для вероятностных характеристик:
    \[
      \alert<1>{\frac{\partial p(t, x_1, x_2)}{\partial t}} = - \sum_{i=1}^2 \frac{\partial}{\partial x_i} [\mu_i(t, x_1, x_2)p(t, x_1, x_2)] +
      \]
     \[
      + \frac{1}{2} \sum_{i=1}^2 \sum_{j=1}^2 \frac{\partial^2}{\partial x_i \partial x_j}\left[ \sum_{i=1}^2 b_{ij}(t, x_1, x_2)b_{ji}(t, x_1, x_2)p(t, x_1, x_2)\right]
    \]
\end{frame}



\begin{frame}

            \[\alert<2>{p(t, x_1, x_2)}+\alert<3>{\Delta t \sum_{n=1}^{10} T_i} = \alert<1>{p(t + \Delta t, x_1, x_2)}\]

\end{frame}

\begin{frame}
  Общая структура $T_i$
  \newline
    \setbeamercolor{itemize item}{fg=white}
    \begin{itemize}
        \item<1-3> {
            \(T_1(t, x_1, x_2) + T_2(t, x_1, x_2) = \alert<2>{p(t, x_1, x_2)}\alert<3>{(1 - d_1 (t, x_1, x_2) - b_1 (t, x_1, x_2) - d_2 (t, x_1, x_2) - b_2 (t, x_1, x_2)- m_{12}(t, x_1, x_2) - m_{21}(t, x_1, x_2) - m_{22}(t, x_1, x_2) - m_{11}(t, x_1, x_2))}\)
        }
        \item<4-6> {
            $T_3(t, x_1 + \lambda_1, x_2) = \alert<5>{p(t, x_1 + \lambda_1, x_2)}\alert<6>{d_1 (t, x_1 + \lambda_1, x_2)}$
        }
    \end{itemize}
\end{frame}

\begin{frame}
    Разложение в ряд Тейлора слагаемых $T_i$
    \newline
    \setbeamercolor{itemize item}{fg=white}
    \begin{itemize}
        \item<1-1> $$T_3 \approx pd_1 + \frac{\partial(pd_1)}{\partial x_1} \lambda_1 + \frac{1}{2}\frac{\partial^2(pd_1)}{\partial x_1^2} \lambda_1^2$$
        \item<2-2> {$$T_7 \approx pm_{12} + \frac{\partial(pm_{12})}{\partial x_1} \lambda_1 - \frac{\partial(pm_{12})}{\partial x_2} \lambda_2 +$$$$+ \frac{1}{2} \sum_{i=1}^2 \sum_{j=1}^2 (-1)^{i+j}\frac{\partial^2(pm_{12})}{\partial x_1 \partial x_2} \lambda_i \lambda_j$$}
    \end{itemize}
\end{frame}

\begin{frame}

    \Wider[4em]
    {\small
        $$\alert<2>{p(t + \Delta t, x_1, x_2) - p(t, x_1, x_2)} = p(1 - d_1 - b_1 - d_2 - b_2 - m_{12} - m_{21} - m_{22} - m_{11}) +$$ $$ + p\left(d_1 + b_1 + b_2 + d_2 + m_{12} + m_{21} + m_{11} + m_{22}\right) + $$ $$+ \left(\frac{\partial(pd_1)}{\partial x_1} - \frac{\partial(pb_1)}{\partial x_1} + \frac{\partial(pm_{12})}{\partial x_1} - \frac{\partial(pm_{21})}{\partial x_1} + \frac{\partial(pm_{11})}{\partial x_1} - \frac{\partial(pm_{22})}{\partial x_1}\right) \lambda_1 + $$ $$- \left(\frac{\partial(pb_2)}{\partial x_2} - \frac{\partial(pd_2)}{\partial x_2} - \frac{\partial(pm_{12})}{\partial x_2} + \frac{\partial(pm_{21})}{\partial x_2}  + \frac{\partial(pm_{11})}{\partial x_2}  - \frac{\partial(pm_{22})}{\partial x_2}\right) \lambda_2 + $$ $$ + \frac{1}{2}\left(\frac{\partial^2(pd_1)}{\partial x_1^2} \lambda_1^2 + \frac{\partial^2(pb_1)}{\partial x_1^2} \lambda_1^2 + \frac{\partial^2(pb_2)}{\partial x_2^2} \lambda_2^2 + \frac{\partial^2(pd_2)}{\partial x_2^2} \lambda_2^2 \right) +$$ $$+ \frac{1}{2}\left(\sum_{i=1}^2 \sum_{j=1}^2  (-1)^{i+j}\frac{\partial^2(pm_{12})}{\partial x_1 \partial x_2}  + \frac{\partial^2(pm_{21})}{\partial x_1 \partial x_2} + \frac{\partial^2(pm_{11})}{\partial x_1 \partial x_2}  + \frac{\partial^2(pm_{22})}{\partial x_1 \partial x_2} \right) \lambda_i \lambda_j$$
    }
\end{frame}

\begin{frame}
    \begin{table}
    \begin{center}
    \begin{tabular}{|c|c|}
    \hline
    \textbf{Изменение значений состояний} & \textbf{Вероятность}\\
    \hline
    $\Delta S^1 = [- \lambda_1, 0]^T$ &  $p_1=d_1 (t, S_1, S_2) \Delta t$ \\
    $\Delta S^2 = [ \lambda_1, 0]^T$ &  $p_2=b_1 (t, S_1, S_2) \Delta t$ \\
    $\Delta S^3 = [ 0, - \lambda_2]^T$ &  $p_3=d_2 (t, S_1, S_2) \Delta t$ \\
    $\Delta S^4 = [ 0, \lambda_2]^T$ &  $p_4=b_2 (t, S_1, S_2) \Delta t$ \\
    $\Delta S^5 = [- \lambda_1, \lambda_2]^T$ &  $p_5=m_{12} (t, S_1, S_2) \Delta t$ \\
    $\Delta S^6 = [ \lambda_1,- \lambda_2]^T$ &  $p_6=m_{21} (t, S_1, S_2) \Delta t$ \\
    $\Delta S^7 = [- \lambda_1,- \lambda_2]^T$ &  $p_7=m_{11} (t, S_1, S_2) \Delta t$ \\
    $\Delta S^8 = [ \lambda_1, \lambda_2]^T$ &  $p_8=m_{22} (t, S_1, S_2) \Delta t$ \\
    $\Delta S^9 = [0, 0]^T$ &  $p_9= 1 - \sum_{n=1}^8 p_i \Delta t$ \\
    \hline

    \end{tabular}
    \end{center}
    \end{table}
\end{frame}

\begin{frame}
Математическое ожидание  $\mathbb{E}(\Delta S)$:

\[
    \mathbb{E}(\Delta S) = \sum_{i=1}^9 p_i \Delta S^i =
        \begin{bmatrix}
            (-d_1+ b_1 - m_{12} + m_{21} + m_{22} - m_{11}) \lambda_1 \\
            (-d_2+ b_2 + m_{12} - m_{21} + m_{22} - m_{11}) \lambda_2 \\
        \end{bmatrix}\Delta t
\]


\small
    Ковариационная матрица:
    \[
        \mathbb{E}(\Delta S (\Delta S)^T) = \sum_{i=1}^9 p_j (\Delta S^j \Delta S^j)^T =
        \]\[\begin{bmatrix}
            (d_1+ b_1 + m_\alpha)\lambda_1^2 & (- m_{12} - m_{21} + m_{22} + m_{11}) \lambda_1 \lambda_1 \\
            (- m_{12} - m_{21} + m_{22} + m_{11}) \lambda_1 \lambda_1 &  (d_2+ b_2 + m_\alpha)\lambda_2^2 \\
        \end{bmatrix}
        \Delta t
    \]
\newline
где \[m_\alpha = m_{12} + m_{21} + m_{22} + m_{11}\]

\end{frame}

\begin{frame}
Уравнение для вероятностных характеристик:
    \[
      \frac{\partial p(t, x_1, x_2)}{\partial t} = - \sum_{i=1}^2 \frac{\partial}{\partial x_i} [\mu_i(t, x_1, x_2)p(t, x_1, x_2)] +
      \]
     \[
      + \frac{1}{2} \sum_{i=1}^2 \sum_{j=1}^2 \frac{\partial^2}{\partial x_i \partial x_j}\left[ \sum_{i=1}^2 b_{ij}(t, x_1, x_2)b_{ji}(t, x_1, x_2)p(t, x_1, x_2)\right]
    \]
\end{frame}

\begin{frame}
    Стохастическое дифференциальное уравнение --- это уравнение вида:

\[dS(t) = \beta(t, S(t)) S(t) dt + \gamma(t, S(t)) S(t) dW(t),\]

%$$t \in [0, T]$$
%$$t \ge s$$
%$$S(0) = S_0$$

\end{frame}

\begin{frame}
В интегральном виде:
  \[S(T) = S(t) + \int_{0}^{t} \beta(t, S(t)) S(t) dt + \int_{0}^{t} \gamma(t, S(t)) S(t) dW(t)\]
\end{frame}

\begin{frame}
  \[dS(t) = aS(t)dt + \sigma S(t)dW(t)\]
  \uncover<2->{\[S_t = S_0 \exp\left((a - \frac{1}{2}\sigma^2)t + \sigma W_t\right)\]}
%Где $a(t)$, $b(t)$, $\gamma(t)$ -- детерминированные, то есть не зависящие от случайности функции.
\end{frame}

\begin{frame}

\end{frame}

\end{document} 